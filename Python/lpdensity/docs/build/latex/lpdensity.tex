%% Generated by Sphinx.
\def\sphinxdocclass{report}
\documentclass[letterpaper,10pt,english]{sphinxmanual}
\ifdefined\pdfpxdimen
   \let\sphinxpxdimen\pdfpxdimen\else\newdimen\sphinxpxdimen
\fi \sphinxpxdimen=.75bp\relax
\ifdefined\pdfimageresolution
    \pdfimageresolution= \numexpr \dimexpr1in\relax/\sphinxpxdimen\relax
\fi
%% let collapsible pdf bookmarks panel have high depth per default
\PassOptionsToPackage{bookmarksdepth=5}{hyperref}

\PassOptionsToPackage{warn}{textcomp}
\usepackage[utf8]{inputenc}
\ifdefined\DeclareUnicodeCharacter
% support both utf8 and utf8x syntaxes
  \ifdefined\DeclareUnicodeCharacterAsOptional
    \def\sphinxDUC#1{\DeclareUnicodeCharacter{"#1}}
  \else
    \let\sphinxDUC\DeclareUnicodeCharacter
  \fi
  \sphinxDUC{00A0}{\nobreakspace}
  \sphinxDUC{2500}{\sphinxunichar{2500}}
  \sphinxDUC{2502}{\sphinxunichar{2502}}
  \sphinxDUC{2514}{\sphinxunichar{2514}}
  \sphinxDUC{251C}{\sphinxunichar{251C}}
  \sphinxDUC{2572}{\textbackslash}
\fi
\usepackage{cmap}
\usepackage[T1]{fontenc}
\usepackage{amsmath,amssymb,amstext}
\usepackage{babel}



\usepackage{tgtermes}
\usepackage{tgheros}
\renewcommand{\ttdefault}{txtt}



\usepackage[Bjarne]{fncychap}
\usepackage{sphinx}

\fvset{fontsize=auto}
\usepackage{geometry}


% Include hyperref last.
\usepackage{hyperref}
% Fix anchor placement for figures with captions.
\usepackage{hypcap}% it must be loaded after hyperref.
% Set up styles of URL: it should be placed after hyperref.
\urlstyle{same}

\addto\captionsenglish{\renewcommand{\contentsname}{Contents:}}

\usepackage{sphinxmessages}
\setcounter{tocdepth}{1}



\title{lpdensity}
\date{May 27, 2022}
\release{1.0.3}
\author{Rajita Chandak}
\newcommand{\sphinxlogo}{\vbox{}}
\renewcommand{\releasename}{Release}
\makeindex
\begin{document}

\pagestyle{empty}
\sphinxmaketitle
\pagestyle{plain}
\sphinxtableofcontents
\pagestyle{normal}
\phantomsection\label{\detokenize{index::doc}}


\sphinxAtStartPar
\sphinxcode{\sphinxupquote{lpdensity}} implements the local polynomial regression based density (and derivatives) estimator and bandwidth selection tools proposed in Cattaneo, Jansson and Ma (2020).  Robust bias\sphinxhyphen{}corrected inference methods,
both pointwise (confidence intervals) and uniform (confidence bands), are also implemented
following the results in Cattaneo, Jansson and Ma (2020, 2021a).
See Cattaneo, Jansson and Ma (2021b) for more implementation details and illustrations.

\sphinxAtStartPar
Install the \sphinxcode{\sphinxupquote{lpdensity}} package by running
\sphinxcode{\sphinxupquote{pip install lpdensity}}.

\sphinxAtStartPar
Import density estimation and bandwidth selection functions by running the following lines

\begin{sphinxVerbatim}[commandchars=\\\{\}]
\PYG{g+gp}{\PYGZgt{}\PYGZgt{}\PYGZgt{} }\PYG{k+kn}{from} \PYG{n+nn}{lpdensity} \PYG{k+kn}{import} \PYG{n}{lpdensity}
\end{sphinxVerbatim}

\begin{sphinxVerbatim}[commandchars=\\\{\}]
\PYG{g+gp}{\PYGZgt{}\PYGZgt{}\PYGZgt{} }\PYG{k+kn}{from} \PYG{n+nn}{lpdensity} \PYG{k+kn}{import} \PYG{n}{lpbwdensity}
\end{sphinxVerbatim}

\sphinxAtStartPar
Related \sphinxcode{\sphinxupquote{Python}}, \sphinxcode{\sphinxupquote{R}} and \sphinxcode{\sphinxupquote{Stata}} packages useful for nonparametric estimation and inference are
available at \sphinxurl{https://nppackages.github.io/}.

\sphinxAtStartPar
For source code and replication files, visit the \sphinxhref{https://github.com/nppackages/lpdensity/}{lpdensity repository}.


\chapter{References}
\label{\detokenize{index:references}}
\sphinxAtStartPar
Calonico, S., M. D. Cattaneo, and M. H. Farrell. 2018.
\sphinxhref{https://nppackages.github.io/references/Calonico-Cattaneo-Farrell\_2018\_JASA.pdf}{On the Effect of Bias Estimation on Coverage Accuracy in Nonparametric Inference}
\sphinxstyleemphasis{Journal of the American Statistical Association}, 113(522): 767\sphinxhyphen{}779.

\sphinxAtStartPar
Calonico, S., M. D. Cattaneo, and M. H. Farrell. 2020.
\sphinxhref{https://nppackages.github.io/references/Calonico-Cattaneo-Farrell\_2020\_CEopt.pdf}{Coverage Error Optimal Confidence Intervals for Local Polynomial Regression}
. Working paper.

\sphinxAtStartPar
Cattaneo, M. D., M. Jansson, and X. Ma. 2020.
\sphinxhref{https://nppackages.github.io/references/Cattaneo-Jansson-Ma\_2020\_JASA.pdf}{Simple Local Polynomial Density Estimators}.
\sphinxstyleemphasis{Journal of the American Statistical Association}, 115(531): 1449\sphinxhyphen{}1455.

\sphinxAtStartPar
Cattaneo, M. D., M. Jansson, and X. Ma. 2021a.
\sphinxhref{https://nppackages.github.io/references/Cattaneo-Jansson-Ma\_2021\_JoE.pdf}{Local Regression Distribution Estimators}
\sphinxstyleemphasis{Journal of Econometrics}, forthcoming.

\sphinxAtStartPar
Cattaneo, M. D., M. Jansson, and X. Ma. 2021b.
\sphinxhref{https://nppackages.github.io/references/Cattaneo-Jansson-Ma\_2021\_JSS.pdf}{lpdensity: Local Polynomial Density Estimation and Inference}
\sphinxstyleemphasis{Journal of Statistical Software}, forthcoming.


\chapter{Authors}
\label{\detokenize{index:authors}}
\sphinxAtStartPar
Matias D. Cattaneo, Princeton University. (\sphinxhref{mailto:cattaneo@princeton.edu}{cattaneo@princeton.edu}).

\sphinxAtStartPar
Rajita Chandak (maintainer), Princeton University. (\sphinxhref{mailto:rchandak@princeton.edu}{rchandak@princeton.edu}).

\sphinxAtStartPar
Michael Jansson, University of California Berkeley. (\sphinxhref{mailto:mjansson@econ.berkeley.edu}{mjansson@econ.berkeley.edu}).

\sphinxAtStartPar
Xinwei Ma (maintainer), University of California San Diego. (\sphinxhref{mailto:x1ma@ucsd.edu}{x1ma@ucsd.edu}).

\sphinxstepscope


\section{lpdensity}
\label{\detokenize{modules:lpdensity}}\label{\detokenize{modules::doc}}
\sphinxstepscope


\subsection{lpdensity}
\label{\detokenize{lpdensity:lpdensity}}\label{\detokenize{lpdensity::doc}}
\sphinxAtStartPar
Local Polynomial Density Estimation and Inference


\subsubsection{Description}
\label{\detokenize{lpdensity:description}}
\sphinxAtStartPar
\sphinxcode{\sphinxupquote{lpdensity}} implements the local polynomial regression based density (and derivatives) estimator proposed in Cattaneo, Jansson and Ma (2020).  Robust bias\sphinxhyphen{}corrected inference methods,
both pointwise (confidence intervals) and uniform (confidence bands), are also implemented
following the results in Cattaneo, Jansson and Ma (2020, 2021a).
See Cattaneo, Jansson and Ma (2021b) for more implementation details and illustrations.

\sphinxAtStartPar
Companion command: \sphinxcode{\sphinxupquote{lpbwdensity}} for bandwidth selection.


\subsubsection{Details}
\label{\detokenize{lpdensity:details}}
\sphinxAtStartPar
Bias correction is only used for the construction of confidence intervals/bands, but not for point
estimation. The point estimates, denoted by \sphinxstyleemphasis{f\_p}, are constructed using local polynomial estimates
of order \sphinxstyleemphasis{p}, while the centering of the confidence intervals/bands, denoted by \sphinxstyleemphasis{f\_q}, are constructed
using local polynomial estimates of order \sphinxstyleemphasis{q}. The confidence intervals/bands take the form:
\sphinxstyleemphasis{{[}f\_q \sphinxhyphen{} cv * SE(f\_q) , f\_q + cv * SE(f\_q){]}}, where \sphinxstyleemphasis{cv} denotes the appropriate critical value and \sphinxstyleemphasis{SE(f\_q)}
denotes an standard error estimate for the centering of the confidence interval/band. As a result,
the confidence intervals/bands may not be centered at the point estimates because they have been bias\sphinxhyphen{}corrected.
Setting \sphinxstyleemphasis{q} and \sphinxstyleemphasis{p} to be equal results on centered at the point estimate confidence intervals/bands,
but requires undersmoothing for valid inference (i.e., (I)MSE\sphinxhyphen{}optimal bandwdith for the density point estimator
cannot be used). Hence the bandwidth would need to be specified manually when \sphinxstyleemphasis{q=p}, and the
point estimates will not be (I)MSE optimal. See Cattaneo, Jansson and Ma (2020a, 2020b) for details, and also
Calonico, Cattaneo, and Farrell (2018, 2020) for robust bias correction methods.

\sphinxAtStartPar
Sometimes the density point estimates may lie outside of the confidence intervals/bands, which can happen
if the underlying distribution exhibits high curvature at some evaluation point(s). One possible solution
in this case is to increase the polynomial order \sphinxstyleemphasis{p} or to employ a smaller bandwidth.


\subsubsection{References}
\label{\detokenize{lpdensity:references}}
\sphinxAtStartPar
Calonico, S., M. D. Cattaneo, and M. H. Farrell. 2018.
\sphinxhref{https://nppackages.github.io/references/Calonico-Cattaneo-Farrell\_2018\_JASA.pdf}{On the Effect of Bias Estimation on Coverage Accuracy in Nonparametric Inference}
\sphinxstyleemphasis{Journal of the American Statistical Association}, 113(522): 767\sphinxhyphen{}779.

\sphinxAtStartPar
Calonico, S., M. D. Cattaneo, and M. H. Farrell. 2020.
\sphinxhref{https://nppackages.github.io/references/Calonico-Cattaneo-Farrell\_2020\_CEopt.pdf}{Coverage Error Optimal Confidence Intervals for Local Polynomial Regression}
. Working paper.

\sphinxAtStartPar
Cattaneo, M. D., M. Jansson, and X. Ma. 2020.
\sphinxhref{https://nppackages.github.io/references/Cattaneo-Jansson-Ma\_2020\_JASA.pdf}{Simple Local Polynomial Density Estimators}.
\sphinxstyleemphasis{Journal of the American Statistical Association}, 115(531): 1449\sphinxhyphen{}1455.

\sphinxAtStartPar
Cattaneo, M. D., M. Jansson, and X. Ma. 2021a.
\sphinxhref{https://nppackages.github.io/references/Cattaneo-Jansson-Ma\_2021\_JoE.pdf}{Local Regression Distribution Estimators}
\sphinxstyleemphasis{Journal of Econometrics}, forthcoming.

\sphinxAtStartPar
Cattaneo, M. D., M. Jansson, and X. Ma. 2021b.
\sphinxhref{https://nppackages.github.io/references/Cattaneo-Jansson-Ma\_2021\_JSS.pdf}{lpdensity: Local Polynomial Density Estimation and Inference}
\sphinxstyleemphasis{Journal of Statistical Software}, forthcoming.


\subsubsection{Authors}
\label{\detokenize{lpdensity:authors}}
\sphinxAtStartPar
Matias D. Cattaneo, Princeton University. (\sphinxhref{mailto:cattaneo@princeton.edu}{cattaneo@princeton.edu}).

\sphinxAtStartPar
Rajita Chandak (maintainer), Princeton University. (\sphinxhref{mailto:rchandak@princeton.edu}{rchandak@princeton.edu}).

\sphinxAtStartPar
Michael Jansson, University of California Berkeley. (\sphinxhref{mailto:mjansson@econ.berkeley.edu}{mjansson@econ.berkeley.edu}).

\sphinxAtStartPar
Xinwei Ma (maintainer), University of California San Diego. (\sphinxhref{mailto:x1ma@ucsd.edu}{x1ma@ucsd.edu}).

\phantomsection\label{\detokenize{lpdensity:module-lpdensity.lpdensity}}\index{module@\spxentry{module}!lpdensity.lpdensity@\spxentry{lpdensity.lpdensity}}\index{lpdensity.lpdensity@\spxentry{lpdensity.lpdensity}!module@\spxentry{module}}\index{lpdensity() (in module lpdensity.lpdensity)@\spxentry{lpdensity()}\spxextra{in module lpdensity.lpdensity}}

\begin{fulllineitems}
\phantomsection\label{\detokenize{lpdensity:lpdensity.lpdensity.lpdensity}}
\pysigstartsignatures
\pysiglinewithargsret{\sphinxcode{\sphinxupquote{lpdensity.lpdensity.}}\sphinxbfcode{\sphinxupquote{lpdensity}}}{\emph{\DUrole{n}{data}}, \emph{\DUrole{n}{grid}\DUrole{o}{=}\DUrole{default_value}{None}}, \emph{\DUrole{n}{bw}\DUrole{o}{=}\DUrole{default_value}{None}}, \emph{\DUrole{n}{p}\DUrole{o}{=}\DUrole{default_value}{None}}, \emph{\DUrole{n}{q}\DUrole{o}{=}\DUrole{default_value}{None}}, \emph{\DUrole{n}{v}\DUrole{o}{=}\DUrole{default_value}{None}}, \emph{\DUrole{n}{kernel}\DUrole{o}{=}\DUrole{default_value}{\textquotesingle{}triangular\textquotesingle{}}}, \emph{\DUrole{n}{scale}\DUrole{o}{=}\DUrole{default_value}{None}}, \emph{\DUrole{n}{massPoints}\DUrole{o}{=}\DUrole{default_value}{True}}, \emph{\DUrole{n}{bwselect}\DUrole{o}{=}\DUrole{default_value}{\textquotesingle{}mse\sphinxhyphen{}dpi\textquotesingle{}}}, \emph{\DUrole{n}{stdVar}\DUrole{o}{=}\DUrole{default_value}{True}}, \emph{\DUrole{n}{regularize}\DUrole{o}{=}\DUrole{default_value}{True}}, \emph{\DUrole{n}{nLocalMin}\DUrole{o}{=}\DUrole{default_value}{None}}, \emph{\DUrole{n}{nUniqueMin}\DUrole{o}{=}\DUrole{default_value}{None}}, \emph{\DUrole{n}{Cweights}\DUrole{o}{=}\DUrole{default_value}{None}}, \emph{\DUrole{n}{Pweights}\DUrole{o}{=}\DUrole{default_value}{None}}}{}
\pysigstopsignatures\begin{quote}\begin{description}
\item[{Parameters}] \leavevmode\begin{itemize}
\item {} 
\sphinxAtStartPar
\sphinxstyleliteralstrong{\sphinxupquote{data}} (\sphinxstyleliteralemphasis{\sphinxupquote{vector}}) \textendash{} Numeric vector or one dimensional matrix/data frame, the raw data.

\item {} 
\sphinxAtStartPar
\sphinxstyleliteralstrong{\sphinxupquote{grid}} (\sphinxstyleliteralemphasis{\sphinxupquote{number}}\sphinxstyleliteralemphasis{\sphinxupquote{ or }}\sphinxstyleliteralemphasis{\sphinxupquote{vector}}) \textendash{} Numeric, specifies the grid of evaluation points. When set to default, grid points will be chosen as 0.05\sphinxhyphen{}0.95 percentiles of the data, with a step size of 0.05.

\item {} 
\sphinxAtStartPar
\sphinxstyleliteralstrong{\sphinxupquote{bw}} (\sphinxstyleliteralemphasis{\sphinxupquote{number}}\sphinxstyleliteralemphasis{\sphinxupquote{ or }}\sphinxstyleliteralemphasis{\sphinxupquote{vector}}) \textendash{} Numeric, specifies the bandwidth used for estimation. Can be (1) a positive scalar (common bandwidth for all grid points); or (2) a positive numeric vector specifying bandwidths for each grid point (should be the same length as \sphinxstyleemphasis{grid}).

\item {} 
\sphinxAtStartPar
\sphinxstyleliteralstrong{\sphinxupquote{p}} (\sphinxstyleliteralemphasis{\sphinxupquote{int}}) \textendash{} Nonnegative integer, specifies the order of the local polynomial used to construct point estimates. (Default is \sphinxstyleemphasis{2}.)

\item {} 
\sphinxAtStartPar
\sphinxstyleliteralstrong{\sphinxupquote{q}} (\sphinxstyleliteralemphasis{\sphinxupquote{int}}) \textendash{} Nonnegative integer, specifies the order of the local polynomial used to construct confidence intervals/bands (a.k.a. the bias correction order). Default is \sphinxstyleemphasis{p+1}. When set to be the same as \sphinxstyleemphasis{p}, no bias correction will be performed. Otherwise it should be strictly larger than \sphinxstyleemphasis{p}.

\item {} 
\sphinxAtStartPar
\sphinxstyleliteralstrong{\sphinxupquote{v}} (\sphinxstyleliteralemphasis{\sphinxupquote{int}}) \textendash{} Nonnegative integer, specifies the derivative of the distribution function to be estimated. \sphinxstyleemphasis{0} for the distribution function, \sphinxstyleemphasis{1} (default) for the density funtion, etc.

\item {} 
\sphinxAtStartPar
\sphinxstyleliteralstrong{\sphinxupquote{kernel}} (\sphinxstyleliteralemphasis{\sphinxupquote{string}}) \textendash{} Specifies the kernel function, should be one of \sphinxstyleemphasis{“triangular”}, \sphinxstyleemphasis{“uniform”}, and \sphinxstyleemphasis{“epanechnikov”}.

\item {} 
\sphinxAtStartPar
\sphinxstyleliteralstrong{\sphinxupquote{scale}} (\sphinxstyleliteralemphasis{\sphinxupquote{number}}) \textendash{} Numeric, specifies how estimates are scaled. For example, setting this parameter to 0.5 will scale down both the point estimates and standard errors by half. Default is \sphinxstyleemphasis{1}. This parameter is useful if only part of the sample is employed for estimation, and should not be confused with \sphinxstyleemphasis{Cweights} or \sphinxstyleemphasis{Pweights}.

\item {} 
\sphinxAtStartPar
\sphinxstyleliteralstrong{\sphinxupquote{massPoints}} (\sphinxstyleliteralemphasis{\sphinxupquote{boolean}}) \textendash{} \sphinxstyleemphasis{True} (default) or \sphinxstyleemphasis{False}, specifies whether point estimates and standard errors should be adjusted if there are mass points in the data.

\item {} 
\sphinxAtStartPar
\sphinxstyleliteralstrong{\sphinxupquote{bwselect}} (\sphinxstyleliteralemphasis{\sphinxupquote{string}}) \textendash{} String, specifies the method for data\sphinxhyphen{}driven bandwidth selection. This option will be ignored if \sphinxstyleemphasis{bw} is provided. Options are (1) \sphinxstyleemphasis{“mse\sphinxhyphen{}dpi”} (default, mean squared error\sphinxhyphen{}optimal bandwidth selected for each grid point); (2) \sphinxstyleemphasis{“imse\sphinxhyphen{}dpi”} (integrated MSE\sphinxhyphen{}optimal bandwidth, common for all grid points); (3) \sphinxstyleemphasis{“mse\sphinxhyphen{}rot”} (rule\sphinxhyphen{}of\sphinxhyphen{}thumb bandwidth with Gaussian reference model); and (4) \sphinxstyleemphasis{“imse\sphinxhyphen{}rot”} (integrated rule\sphinxhyphen{}of\sphinxhyphen{}thumb bandwidth with Gaussian reference model).

\item {} 
\sphinxAtStartPar
\sphinxstyleliteralstrong{\sphinxupquote{stdVar}} (\sphinxstyleliteralemphasis{\sphinxupquote{boolean}}) \textendash{} \sphinxstyleemphasis{True} (default) or \sphinxstyleemphasis{False}, specifies whether the data should be standardized for bandwidth selection.

\item {} 
\sphinxAtStartPar
\sphinxstyleliteralstrong{\sphinxupquote{regularize}} (\sphinxstyleliteralemphasis{\sphinxupquote{boolean}}) \textendash{} \sphinxstyleemphasis{True} (default) or \sphinxstyleemphasis{False}, specifies whether the bandwidth should be regularized. When set to \sphinxstyleemphasis{True}, the bandwidth is chosen such that the local region includes at least \sphinxstyleemphasis{nLocalMin} observations and at least \sphinxstyleemphasis{nUniqueMin} unique observations.

\item {} 
\sphinxAtStartPar
\sphinxstyleliteralstrong{\sphinxupquote{nLocalMin}} (\sphinxstyleliteralemphasis{\sphinxupquote{int}}) \textendash{} Nonnegative integer, specifies the minimum number of observations in each local neighborhood. This option will be ignored if \sphinxstyleemphasis{regularize=False}. Default is \sphinxstyleemphasis{20+p+1}.

\item {} 
\sphinxAtStartPar
\sphinxstyleliteralstrong{\sphinxupquote{nUniqueMin}} (\sphinxstyleliteralemphasis{\sphinxupquote{int}}) \textendash{} Nonnegative integer, specifies the minimum number of unique observations in each local neighborhood. This option will be ignored if \sphinxstyleemphasis{regularize=False}. Default is \sphinxstyleemphasis{20+p+1}.

\item {} 
\sphinxAtStartPar
\sphinxstyleliteralstrong{\sphinxupquote{Cweights}} (\sphinxstyleliteralemphasis{\sphinxupquote{numeric vector}}) \textendash{} Numeric, specifies the weights used for counterfactual distribution construction. Should have the same length as the data.

\item {} 
\sphinxAtStartPar
\sphinxstyleliteralstrong{\sphinxupquote{Pweights}} (\sphinxstyleliteralemphasis{\sphinxupquote{numeric vector}}) \textendash{} Numeric, specifies the weights used in sampling. Should have the same length as the data.

\end{itemize}

\item[{Returns}] \leavevmode
\sphinxAtStartPar
\begin{itemize}
\item {} 
\sphinxAtStartPar
\sphinxstyleemphasis{Estimate} \textendash{} A matrix containing (1) \sphinxstyleemphasis{grid} (grid points), (2) \sphinxstyleemphasis{bw} (bandwidths), (3) \sphinxstyleemphasis{nh} (number of observations in each local neighborhood), (4) \sphinxstyleemphasis{nhu} (number of unique observations in each local neighborhood), (5) \sphinxstyleemphasis{f\_p} (point estimates with p\sphinxhyphen{}th order local polynomial), (6) \sphinxstyleemphasis{f\_q} (point estimates with q\sphinxhyphen{}th order local polynomial, only if option \sphinxstyleemphasis{q} is nonzero), (7) \sphinxstyleemphasis{se\_p} (standard error corresponding to \sphinxstyleemphasis{f\_p}), and (8) \sphinxstyleemphasis{se\_q} (standard error corresponding to \sphinxstyleemphasis{f\_q}).

\item {} 
\sphinxAtStartPar
\sphinxstyleemphasis{CovMat\_p} \textendash{} The variance\sphinxhyphen{}covariance matrix corresponding to \sphinxstyleemphasis{f\_p}.

\item {} 
\sphinxAtStartPar
\sphinxstyleemphasis{CovMat\_q} \textendash{} The variance\sphinxhyphen{}covariance matrix corresponding to \sphinxstyleemphasis{f\_q}.

\end{itemize}


\end{description}\end{quote}

\end{fulllineitems}

\index{lpdensity\_output (class in lpdensity.lpdensity)@\spxentry{lpdensity\_output}\spxextra{class in lpdensity.lpdensity}}

\begin{fulllineitems}
\phantomsection\label{\detokenize{lpdensity:lpdensity.lpdensity.lpdensity_output}}
\pysigstartsignatures
\pysiglinewithargsret{\sphinxbfcode{\sphinxupquote{class\DUrole{w}{  }}}\sphinxcode{\sphinxupquote{lpdensity.lpdensity.}}\sphinxbfcode{\sphinxupquote{lpdensity\_output}}}{\emph{\DUrole{n}{Estimate}}, \emph{\DUrole{n}{CovMat\_p}}, \emph{\DUrole{n}{CovMat\_q}}, \emph{\DUrole{n}{p}}, \emph{\DUrole{n}{q}}, \emph{\DUrole{n}{v}}, \emph{\DUrole{n}{kernel}}, \emph{\DUrole{n}{scale}}, \emph{\DUrole{n}{massPoints}}, \emph{\DUrole{n}{n}}, \emph{\DUrole{n}{ng}}, \emph{\DUrole{n}{bwselect}}, \emph{\DUrole{n}{stdVar}}, \emph{\DUrole{n}{regularize}}, \emph{\DUrole{n}{nLocalMin}}, \emph{\DUrole{n}{nUniqueMin}}, \emph{\DUrole{n}{data\_min}}, \emph{\DUrole{n}{data\_max}}, \emph{\DUrole{n}{grid\_min}}, \emph{\DUrole{n}{grid\_max}}}{}
\pysigstopsignatures
\sphinxAtStartPar
Class of lpdensity function outputs.

\sphinxAtStartPar
Object type returned by {\hyperref[\detokenize{lpdensity:lpdensity.lpdensity.lpdensity}]{\sphinxcrossref{\sphinxcode{\sphinxupquote{lpdensity()}}}}}.
\index{coef() (lpdensity.lpdensity.lpdensity\_output method)@\spxentry{coef()}\spxextra{lpdensity.lpdensity.lpdensity\_output method}}

\begin{fulllineitems}
\phantomsection\label{\detokenize{lpdensity:lpdensity.lpdensity.lpdensity_output.coef}}
\pysigstartsignatures
\pysiglinewithargsret{\sphinxbfcode{\sphinxupquote{coef}}}{}{}
\pysigstopsignatures
\sphinxAtStartPar
Returns estimate coefficients.

\end{fulllineitems}

\index{confint() (lpdensity.lpdensity.lpdensity\_output method)@\spxentry{confint()}\spxextra{lpdensity.lpdensity.lpdensity\_output method}}

\begin{fulllineitems}
\phantomsection\label{\detokenize{lpdensity:lpdensity.lpdensity.lpdensity_output.confint}}
\pysigstartsignatures
\pysiglinewithargsret{\sphinxbfcode{\sphinxupquote{confint}}}{\emph{\DUrole{n}{alpha}\DUrole{o}{=}\DUrole{default_value}{0.05}}, \emph{\DUrole{n}{CIuniform}\DUrole{o}{=}\DUrole{default_value}{False}}, \emph{\DUrole{n}{CIsimul}\DUrole{o}{=}\DUrole{default_value}{2000}}}{}
\pysigstopsignatures
\sphinxAtStartPar
Returns confindence intervals/bands for prespecified confidence level.
\begin{description}
\item[{alpha}] \leavevmode{[}number{]}
\sphinxAtStartPar
Confindence level, must be between 0 and 1.

\item[{CIuniform}] \leavevmode{[}boolean{]}
\sphinxAtStartPar
Boolean on wehther to construct uniform confidence bands, \sphinxstyleemphasis{True} or \sphinxstyleemphasis{False (default)}.

\item[{CIsimul}] \leavevmode{[}int{]}
\sphinxAtStartPar
Number of simulations used to generate confidence intervals.

\end{description}

\end{fulllineitems}

\index{plot() (lpdensity.lpdensity.lpdensity\_output method)@\spxentry{plot()}\spxextra{lpdensity.lpdensity.lpdensity\_output method}}

\begin{fulllineitems}
\phantomsection\label{\detokenize{lpdensity:lpdensity.lpdensity.lpdensity_output.plot}}
\pysigstartsignatures
\pysiglinewithargsret{\sphinxbfcode{\sphinxupquote{plot}}}{\emph{\DUrole{n}{alpha}\DUrole{o}{=}\DUrole{default_value}{0.05}}, \emph{\DUrole{n}{type}\DUrole{o}{=}\DUrole{default_value}{\textquotesingle{}line\textquotesingle{}}}, \emph{\DUrole{n}{CItype}\DUrole{o}{=}\DUrole{default_value}{\textquotesingle{}region\textquotesingle{}}}, \emph{\DUrole{n}{CIuniform}\DUrole{o}{=}\DUrole{default_value}{False}}, \emph{\DUrole{n}{CIsimul}\DUrole{o}{=}\DUrole{default_value}{2000}}, \emph{\DUrole{n}{hist}\DUrole{o}{=}\DUrole{default_value}{False}}, \emph{\DUrole{n}{histData}\DUrole{o}{=}\DUrole{default_value}{None}}, \emph{\DUrole{n}{histBins}\DUrole{o}{=}\DUrole{default_value}{None}}, \emph{\DUrole{n}{histFillCol}\DUrole{o}{=}\DUrole{default_value}{3}}, \emph{\DUrole{n}{histFillShade}\DUrole{o}{=}\DUrole{default_value}{0.2}}, \emph{\DUrole{n}{histLineCol}\DUrole{o}{=}\DUrole{default_value}{\textquotesingle{}white\textquotesingle{}}}, \emph{\DUrole{n}{title}\DUrole{o}{=}\DUrole{default_value}{None}}, \emph{\DUrole{n}{xlabel}\DUrole{o}{=}\DUrole{default_value}{None}}, \emph{\DUrole{n}{ylabel}\DUrole{o}{=}\DUrole{default_value}{None}}, \emph{\DUrole{n}{CIshade}\DUrole{o}{=}\DUrole{default_value}{0.2}}}{}
\pysigstopsignatures
\sphinxAtStartPar
Method to plot estimate and confidence bands.
Requires ggplot.
\begin{description}
\item[{alpha}] \leavevmode{[}number{]}
\sphinxAtStartPar
Confindence level, must be between 0 and 1.

\item[{CIuniform}] \leavevmode{[}boolean{]}
\sphinxAtStartPar
Boolean on wehther to construct uniform confidence bands, \sphinxstyleemphasis{True} or \sphinxstyleemphasis{False (default)}.

\item[{CIsimul}] \leavevmode{[}int{]}
\sphinxAtStartPar
Number of simulations used to generate confidence intervals.

\item[{type}] \leavevmode{[}string{]}
\sphinxAtStartPar
type of estimate plot, \sphinxstyleemphasis{line (default)}, or \sphinxstyleemphasis{points}, or \sphinxstyleemphasis{all}.

\item[{CItype}] \leavevmode{[}string{]}
\sphinxAtStartPar
type of confidence interval plot, \sphinxstyleemphasis{region (default)}, \sphinxstyleemphasis{lines}, \sphinxstyleemphasis{ebar}, or \sphinxstyleemphasis{all}.

\end{description}

\end{fulllineitems}

\index{vcov() (lpdensity.lpdensity.lpdensity\_output method)@\spxentry{vcov()}\spxextra{lpdensity.lpdensity.lpdensity\_output method}}

\begin{fulllineitems}
\phantomsection\label{\detokenize{lpdensity:lpdensity.lpdensity.lpdensity_output.vcov}}
\pysigstartsignatures
\pysiglinewithargsret{\sphinxbfcode{\sphinxupquote{vcov}}}{}{}
\pysigstopsignatures
\sphinxAtStartPar
Returns estimate standard error and covariance matrices.

\end{fulllineitems}


\end{fulllineitems}



\subsubsection{Example}
\label{\detokenize{lpdensity:example}}
\begin{sphinxVerbatim}[commandchars=\\\{\}]
\PYG{g+gp}{\PYGZgt{}\PYGZgt{}\PYGZgt{} }\PYG{k+kn}{import} \PYG{n+nn}{numpy} \PYG{k}{as} \PYG{n+nn}{np}
\PYG{g+gp}{\PYGZgt{}\PYGZgt{}\PYGZgt{} }\PYG{k+kn}{from} \PYG{n+nn}{lpdensity} \PYG{k+kn}{import} \PYG{n}{lpdensity}
\PYG{g+gp}{\PYGZgt{}\PYGZgt{}\PYGZgt{} }\PYG{n}{data} \PYG{o}{=} \PYG{n}{np}\PYG{o}{.}\PYG{n}{random}\PYG{o}{.}\PYG{n}{normal}\PYG{p}{(}\PYG{l+m+mi}{0}\PYG{p}{,}\PYG{l+m+mi}{1}\PYG{p}{,}\PYG{l+m+mi}{500}\PYG{p}{)}
\PYG{g+gp}{\PYGZgt{}\PYGZgt{}\PYGZgt{} }\PYG{n}{grid} \PYG{o}{=} \PYG{n}{np}\PYG{o}{.}\PYG{n}{linspace}\PYG{p}{(}\PYG{n+nb}{min}\PYG{p}{(}\PYG{n}{data}\PYG{p}{)}\PYG{p}{,} \PYG{n+nb}{max}\PYG{p}{(}\PYG{n}{data}\PYG{p}{)}\PYG{p}{,} \PYG{l+m+mi}{10}\PYG{p}{)}
\PYG{g+gp}{\PYGZgt{}\PYGZgt{}\PYGZgt{} }\PYG{n}{est} \PYG{o}{=} \PYG{n}{lpdensity}\PYG{p}{(}\PYG{n}{data}\PYG{o}{=}\PYG{n}{data}\PYG{p}{,} \PYG{n}{grid}\PYG{o}{=}\PYG{n}{grid}\PYG{p}{)}
\PYG{g+gp}{\PYGZgt{}\PYGZgt{}\PYGZgt{} }\PYG{n+nb}{print}\PYG{p}{(}\PYG{n+nb}{repr}\PYG{p}{(}\PYG{n}{est}\PYG{p}{)}\PYG{p}{)}
\PYG{g+gp}{\PYGZgt{}\PYGZgt{}\PYGZgt{} }\PYG{n}{est}\PYG{o}{.}\PYG{n}{plot}\PYG{p}{(}\PYG{p}{)}
\end{sphinxVerbatim}

\sphinxstepscope


\subsection{lpbwdensity}
\label{\detokenize{lpbwdensity:lpbwdensity}}\label{\detokenize{lpbwdensity::doc}}
\sphinxAtStartPar
Data\sphinxhyphen{}driven Bandwidth Selection for Local Polynomial Density Estimators


\subsubsection{Description}
\label{\detokenize{lpbwdensity:description}}
\sphinxAtStartPar
\sphinxcode{\sphinxupquote{lpbwdensity}} implements the bandwidth selection methods for local
polynomial based density (and derivatives) estimation proposed and studied
in Cattaneo, Jansson and Ma (2020, 2021a).
See Cattaneo, Jansson and Ma (2021b) for more implementation details and illustrations.

\sphinxAtStartPar
Companion command: \sphinxcode{\sphinxupquote{lpdensity}} for estimation and robust bias\sphinxhyphen{}corrected inference.


\subsubsection{References}
\label{\detokenize{lpbwdensity:references}}
\sphinxAtStartPar
Cattaneo, M. D., M. Jansson, and X. Ma. 2020.
\sphinxhref{https://nppackages.github.io/references/Cattaneo-Jansson-Ma\_2020\_JASA.pdf}{Simple Local Polynomial Density Estimators}.
\sphinxstyleemphasis{Journal of the American Statistical Association}, 115(531): 1449\sphinxhyphen{}1455.

\sphinxAtStartPar
Cattaneo, M. D., M. Jansson, and X. Ma. 2021a.
\sphinxhref{https://nppackages.github.io/references/Cattaneo-Jansson-Ma\_2021\_JoE.pdf}{Local Regression Distribution Estimators}
\sphinxstyleemphasis{Journal of Econometrics}, forthcoming.

\sphinxAtStartPar
Cattaneo, M. D., M. Jansson, and X. Ma. 2021b.
\sphinxhref{https://nppackages.github.io/references/Cattaneo-Jansson-Ma\_2021\_JSS.pdf}{lpdensity: Local Polynomial Density Estimation and Inference}
\sphinxstyleemphasis{Journal of Statistical Software}, forthcoming.


\subsubsection{Authors}
\label{\detokenize{lpbwdensity:authors}}
\sphinxAtStartPar
Matias D. Cattaneo, Princeton University. (\sphinxhref{mailto:cattaneo@princeton.edu}{cattaneo@princeton.edu}).

\sphinxAtStartPar
Rajita Chandak (maintainer), Princeton University. (\sphinxhref{mailto:rchandak@princeton.edu}{rchandak@princeton.edu}).

\sphinxAtStartPar
Michael Jansson, University of California Berkeley. (\sphinxhref{mailto:mjansson@econ.berkeley.edu}{mjansson@econ.berkeley.edu}).

\sphinxAtStartPar
Xinwei Ma (maintainer), University of California San Diego. (\sphinxhref{mailto:x1ma@ucsd.edu}{x1ma@ucsd.edu}).

\phantomsection\label{\detokenize{lpbwdensity:module-lpdensity.lpbwdensity}}\index{module@\spxentry{module}!lpdensity.lpbwdensity@\spxentry{lpdensity.lpbwdensity}}\index{lpdensity.lpbwdensity@\spxentry{lpdensity.lpbwdensity}!module@\spxentry{module}}\index{bw\_output (class in lpdensity.lpbwdensity)@\spxentry{bw\_output}\spxextra{class in lpdensity.lpbwdensity}}

\begin{fulllineitems}
\phantomsection\label{\detokenize{lpbwdensity:lpdensity.lpbwdensity.bw_output}}
\pysigstartsignatures
\pysiglinewithargsret{\sphinxbfcode{\sphinxupquote{class\DUrole{w}{  }}}\sphinxcode{\sphinxupquote{lpdensity.lpbwdensity.}}\sphinxbfcode{\sphinxupquote{bw\_output}}}{\emph{\DUrole{n}{BW}}, \emph{\DUrole{n}{bws}}, \emph{\DUrole{n}{p}}, \emph{\DUrole{n}{v}}, \emph{\DUrole{n}{kernel}}, \emph{\DUrole{n}{n}}, \emph{\DUrole{n}{ng}}, \emph{\DUrole{n}{bwselect}}, \emph{\DUrole{n}{massPoints}}, \emph{\DUrole{n}{stdVar}}, \emph{\DUrole{n}{regularize}}, \emph{\DUrole{n}{nLocalMin}}, \emph{\DUrole{n}{nUniqueMin}}, \emph{\DUrole{n}{data\_min}}, \emph{\DUrole{n}{data\_max}}, \emph{\DUrole{n}{grid\_min}}, \emph{\DUrole{n}{grid\_max}}}{}
\pysigstopsignatures
\sphinxAtStartPar
Class of lpbwdensity function outputs.

\sphinxAtStartPar
Object type returned by {\hyperref[\detokenize{lpbwdensity:lpdensity.lpbwdensity.lpbwdensity}]{\sphinxcrossref{\sphinxcode{\sphinxupquote{lpbwdensity()}}}}}.
\index{coef() (lpdensity.lpbwdensity.bw\_output method)@\spxentry{coef()}\spxextra{lpdensity.lpbwdensity.bw\_output method}}

\begin{fulllineitems}
\phantomsection\label{\detokenize{lpbwdensity:lpdensity.lpbwdensity.bw_output.coef}}
\pysigstartsignatures
\pysiglinewithargsret{\sphinxbfcode{\sphinxupquote{coef}}}{}{}
\pysigstopsignatures
\sphinxAtStartPar
Returns estimate of bandwidths.

\end{fulllineitems}


\end{fulllineitems}

\index{lpbwdensity() (in module lpdensity.lpbwdensity)@\spxentry{lpbwdensity()}\spxextra{in module lpdensity.lpbwdensity}}

\begin{fulllineitems}
\phantomsection\label{\detokenize{lpbwdensity:lpdensity.lpbwdensity.lpbwdensity}}
\pysigstartsignatures
\pysiglinewithargsret{\sphinxcode{\sphinxupquote{lpdensity.lpbwdensity.}}\sphinxbfcode{\sphinxupquote{lpbwdensity}}}{\emph{\DUrole{n}{data}}, \emph{\DUrole{n}{grid}\DUrole{o}{=}\DUrole{default_value}{None}}, \emph{\DUrole{n}{p}\DUrole{o}{=}\DUrole{default_value}{None}}, \emph{\DUrole{n}{v}\DUrole{o}{=}\DUrole{default_value}{None}}, \emph{\DUrole{n}{kernel}\DUrole{o}{=}\DUrole{default_value}{\textquotesingle{}triangular\textquotesingle{}}}, \emph{\DUrole{n}{bwselect}\DUrole{o}{=}\DUrole{default_value}{\textquotesingle{}mse\sphinxhyphen{}dpi\textquotesingle{}}}, \emph{\DUrole{n}{massPoints}\DUrole{o}{=}\DUrole{default_value}{True}}, \emph{\DUrole{n}{stdVar}\DUrole{o}{=}\DUrole{default_value}{True}}, \emph{\DUrole{n}{regularize}\DUrole{o}{=}\DUrole{default_value}{True}}, \emph{\DUrole{n}{nLocalMin}\DUrole{o}{=}\DUrole{default_value}{None}}, \emph{\DUrole{n}{nUniqueMin}\DUrole{o}{=}\DUrole{default_value}{None}}, \emph{\DUrole{n}{Cweights}\DUrole{o}{=}\DUrole{default_value}{None}}, \emph{\DUrole{n}{Pweights}\DUrole{o}{=}\DUrole{default_value}{None}}}{}
\pysigstopsignatures\begin{quote}\begin{description}
\item[{Parameters}] \leavevmode\begin{itemize}
\item {} 
\sphinxAtStartPar
\sphinxstyleliteralstrong{\sphinxupquote{data}} (\sphinxstyleliteralemphasis{\sphinxupquote{vector}}) \textendash{} Numeric vector or one dimensional matrix/data frame, the raw data.

\item {} 
\sphinxAtStartPar
\sphinxstyleliteralstrong{\sphinxupquote{grid}} (\sphinxstyleliteralemphasis{\sphinxupquote{vector}}) \textendash{} Numeric, specifies the grid of evaluation points. When set to default, grid points will be chosen as 0.05\sphinxhyphen{}0.95 percentiles of the data, with a step size of 0.05.

\item {} 
\sphinxAtStartPar
\sphinxstyleliteralstrong{\sphinxupquote{p}} (\sphinxstyleliteralemphasis{\sphinxupquote{int}}) \textendash{} Nonnegative integer, specifies the order of the local polynomial used to construct point estimates. (Default is \sphinxstyleemphasis{2}.)

\item {} 
\sphinxAtStartPar
\sphinxstyleliteralstrong{\sphinxupquote{v}} (\sphinxstyleliteralemphasis{\sphinxupquote{int}}) \textendash{} Nonnegative integer, specifies the derivative of the distribution function to be estimated. \sphinxstyleemphasis{0} for the distribution function, \sphinxstyleemphasis{1} (default) for the density funtion, etc.

\item {} 
\sphinxAtStartPar
\sphinxstyleliteralstrong{\sphinxupquote{kernel}} (\sphinxstyleliteralemphasis{\sphinxupquote{string}}) \textendash{} Specifies the kernel function, should be one of \sphinxstyleemphasis{“triangular”}, \sphinxstyleemphasis{“uniform”} or
\sphinxstyleemphasis{“epanechnikov”}.

\item {} 
\sphinxAtStartPar
\sphinxstyleliteralstrong{\sphinxupquote{bwselect}} (\sphinxstyleliteralemphasis{\sphinxupquote{string}}) \textendash{} Specifies the method for data\sphinxhyphen{}driven bandwidth selection. This option will be
ignored if \sphinxstyleemphasis{bw} is provided. Can be (1) \sphinxstyleemphasis{“mse\sphinxhyphen{}dpi”} (default, mean squared error\sphinxhyphen{}optimal
bandwidth selected for each grid point); or (2) \sphinxstyleemphasis{“imse\sphinxhyphen{}dpi”} (integrated MSE\sphinxhyphen{}optimal bandwidth,
common for all grid points); (3) \sphinxstyleemphasis{“mse\sphinxhyphen{}rot”} (rule\sphinxhyphen{}of\sphinxhyphen{}thumb bandwidth with Gaussian
reference model); and (4) \sphinxstyleemphasis{“imse\sphinxhyphen{}rot”} (integrated rule\sphinxhyphen{}of\sphinxhyphen{}thumb bandwidth with Gaussian
reference model).

\item {} 
\sphinxAtStartPar
\sphinxstyleliteralstrong{\sphinxupquote{massPoints}} (\sphinxstyleliteralemphasis{\sphinxupquote{boolean}}) \textendash{} \sphinxstyleemphasis{True} (default) or \sphinxstyleemphasis{False}, specifies whether point estimates and standard errors
should be adjusted if there are mass points in the data.

\item {} 
\sphinxAtStartPar
\sphinxstyleliteralstrong{\sphinxupquote{stdVar}} (\sphinxstyleliteralemphasis{\sphinxupquote{boolean}}) \textendash{} \sphinxstyleemphasis{True} (default) or \sphinxstyleemphasis{False}, specifies whether the data should be standardized for
bandwidth selection.

\item {} 
\sphinxAtStartPar
\sphinxstyleliteralstrong{\sphinxupquote{regularize}} (\sphinxstyleliteralemphasis{\sphinxupquote{boolean}}) \textendash{} \sphinxstyleemphasis{True} (default) or \sphinxstyleemphasis{False}, specifies whether the bandwidth should be
regularized. When set to \sphinxstyleemphasis{True}, the bandwidth is chosen such that the local region includes
at least \sphinxstyleemphasis{nLocalMin} observations and at least \sphinxstyleemphasis{nUniqueMin} unique observations.

\item {} 
\sphinxAtStartPar
\sphinxstyleliteralstrong{\sphinxupquote{nLocalMin}} (\sphinxstyleliteralemphasis{\sphinxupquote{int}}) \textendash{} Nonnegative integer, specifies the minimum number of observations in each local neighborhood. This option
will be ignored if \sphinxstyleemphasis{regularize=False}. Default is \sphinxstyleemphasis{20+p+1}.

\item {} 
\sphinxAtStartPar
\sphinxstyleliteralstrong{\sphinxupquote{nUniqueMin}} (\sphinxstyleliteralemphasis{\sphinxupquote{int}}) \textendash{} Nonnegative integer, specifies the minimum number of unique observations in each local neighborhood. This option
will be ignored if \sphinxstyleemphasis{regularize=False}. Default is \sphinxstyleemphasis{20+p+1}.

\item {} 
\sphinxAtStartPar
\sphinxstyleliteralstrong{\sphinxupquote{Cweights}} (\sphinxstyleliteralemphasis{\sphinxupquote{vector}}) \textendash{} Numeric vector, specifies the weights used
for counterfactual distribution construction. Should have the same length as the data.
This option will be ignored if \sphinxstyleemphasis{bwselect} is \sphinxstyleemphasis{“mse\sphinxhyphen{}rot”} or \sphinxstyleemphasis{“imse\sphinxhyphen{}rot”}.

\item {} 
\sphinxAtStartPar
\sphinxstyleliteralstrong{\sphinxupquote{Pweights}} (\sphinxstyleliteralemphasis{\sphinxupquote{vector}}) \textendash{} Numeric vector, specifies the weights used
in sampling. Should have the same length as the data.
This option will be ignored if \sphinxstyleemphasis{bwselect} is \sphinxstyleemphasis{“mse\sphinxhyphen{}rot”} or \sphinxstyleemphasis{“imse\sphinxhyphen{}rot”}.

\end{itemize}

\item[{Returns}] \leavevmode
\sphinxAtStartPar
\sphinxstylestrong{BW} \textendash{} A {\hyperref[\detokenize{lpbwdensity:lpdensity.lpbwdensity.bw_output}]{\sphinxcrossref{\sphinxcode{\sphinxupquote{bw\_output()}}}}} class object containing a matrix of (1) \sphinxstyleemphasis{grid} (grid point), (2) \sphinxstyleemphasis{bw} (bandwidth),
(3) \sphinxstyleemphasis{nh} (number of observations in each local neighborhood),
(4) \sphinxstyleemphasis{nhu} (number of unique observations in each local neighborhood), and
(5) \sphinxstyleemphasis{opt} additional parameters.

\item[{Return type}] \leavevmode
\sphinxAtStartPar
object

\end{description}\end{quote}

\end{fulllineitems}



\subsubsection{Example}
\label{\detokenize{lpbwdensity:example}}
\begin{sphinxVerbatim}[commandchars=\\\{\}]
\PYG{g+gp}{\PYGZgt{}\PYGZgt{}\PYGZgt{} }\PYG{k+kn}{import} \PYG{n+nn}{numpy} \PYG{k}{as} \PYG{n+nn}{np}
\PYG{g+gp}{\PYGZgt{}\PYGZgt{}\PYGZgt{} }\PYG{k+kn}{from} \PYG{n+nn}{lpdensity} \PYG{k+kn}{import} \PYG{n}{lpbwdensity}
\PYG{g+gp}{\PYGZgt{}\PYGZgt{}\PYGZgt{} }\PYG{n}{data} \PYG{o}{=} \PYG{n}{np}\PYG{o}{.}\PYG{n}{random}\PYG{o}{.}\PYG{n}{normal}\PYG{p}{(}\PYG{l+m+mi}{0}\PYG{p}{,}\PYG{l+m+mi}{1}\PYG{p}{,}\PYG{l+m+mi}{500}\PYG{p}{)}
\PYG{g+gp}{\PYGZgt{}\PYGZgt{}\PYGZgt{} }\PYG{n}{grid} \PYG{o}{=} \PYG{n}{np}\PYG{o}{.}\PYG{n}{linspace}\PYG{p}{(}\PYG{n+nb}{min}\PYG{p}{(}\PYG{n}{data}\PYG{p}{)}\PYG{p}{,} \PYG{n+nb}{max}\PYG{p}{(}\PYG{n}{data}\PYG{p}{)}\PYG{p}{,} \PYG{l+m+mi}{10}\PYG{p}{)}
\PYG{g+gp}{\PYGZgt{}\PYGZgt{}\PYGZgt{} }\PYG{n}{est} \PYG{o}{=} \PYG{n}{lpbwdensity}\PYG{p}{(}\PYG{n}{data}\PYG{o}{=}\PYG{n}{data}\PYG{p}{,} \PYG{n}{grid}\PYG{o}{=}\PYG{n}{grid}\PYG{p}{,} \PYG{n}{bwselect}\PYG{o}{=}\PYG{l+s+s2}{\PYGZdq{}}\PYG{l+s+s2}{mse\PYGZhy{}dpi}\PYG{l+s+s2}{\PYGZdq{}}\PYG{p}{)}
\PYG{g+gp}{\PYGZgt{}\PYGZgt{}\PYGZgt{} }\PYG{n+nb}{print}\PYG{p}{(}\PYG{n+nb}{repr}\PYG{p}{(}\PYG{n}{est}\PYG{p}{)}\PYG{p}{)}
\end{sphinxVerbatim}


\renewcommand{\indexname}{Python Module Index}
\begin{sphinxtheindex}
\let\bigletter\sphinxstyleindexlettergroup
\bigletter{l}
\item\relax\sphinxstyleindexentry{lpdensity.lpbwdensity}\sphinxstyleindexpageref{lpbwdensity:\detokenize{module-lpdensity.lpbwdensity}}
\item\relax\sphinxstyleindexentry{lpdensity.lpdensity}\sphinxstyleindexpageref{lpdensity:\detokenize{module-lpdensity.lpdensity}}
\end{sphinxtheindex}

\renewcommand{\indexname}{Index}
\printindex
\end{document}